%\documentclass[prb,11pt]{revtex4}
%\documentclass[twocolumn]{revtex4}

%\usepackage{amsfonts, amsmath, amssymb, mathrsfs}
%\usepackage{verbatim}
%\usepackage{url}
%\usepackage[pdftex]{graphicx}           % Standard graphics package
%\renewcommand{\thefootnote}{\fnsymbol{footnote}}


%\begin{document}
\section{nPDFs in the high-x region}

Nuclear parton distributions functions (nPDFs) are the first step in understanding the behavior of nuclear matter at the elementary particle level. Moreover they play a crucial role in the determination of the free proton PDFs, as light nuclear targets are routinely used for separating the different partonic flavors in PDFs fits. Notwithstanding their importance, nPDFs are not as well known as the free proton ones, mainly due to two factors. On one side, despite the phenomenal success of HERA in determining the proton structure, no $e+A$ collider has ever existed. On the other side, only a few nuclei have been studied in detail (He, C, Fe) and the data span a limited region of the kinematic space, to the point that the only constrained nuclear distributions are the valence quarks in the mid-x region, assuming the same nuclear modification for the up and down quarks. Furthermore, the parameterizations of the nuclear effects in the neutron are done by assuming the validity of the isospin symmetry, which which is hard to test given the nearly isoscalar nature of most available nuclei.  Less than about one third of the data used in the fits come from heavy nuclei which complicates the possibility of truly separating the nuclear modifications for each flavour. A possible path for doing so would be using charged current (CC) data from DIS, available for iron and lead, as the cross-section depend on different combinations of the PDFs than the neutral current (NC) processes. It has been also suggested that nuclear effects might not be universal and therefore making a truly global fit of the nPDFs would not be possible. Up to now and within experimental uncertainties, NLO fits including NC and CC data have not shown visible tension. Unfortunately these fixed target experiments cover a very limited region of the kinematic space and are lacking in precision, not allowing yet for a conclusive answer.


 Extrapolations for $x < 10^{-3}$ obey the fulfillment of the charge and momentum sum rules and, at mid to high-$x$, the sea quarks and gluon densities are determined by ad-hoc assumptions during the fitting procedure, rather than from actual constraints from the data. 

The unexplored low-$x$ region, dominated by the gluon density, opens the possibility of finding new non-linear phenomena such saturation, and puts to test the applicability range of the linear regime. The other extreme of the kinematic space, the high-$x$ region, is of particular interest as there appears the first measured sign of nuclear effects in high energy collisions: the EMC effect. Moreover, for beyond the Standard Model searches at the LHC, rare high-$x$ gluon initiated events could be enhanced. However the nuclear gluon is difficult to access at high-$x$, and great care has to be put in its determination. Despite the lack of data, the determination of nPDFs in all the kinematic space is of crucial relevance and thus the target of several efforts. Given the fact that sea quarks and gluon densities are tied through the DGLAP evolution equations, studying processes sensitive to either sea quarks or gluons has an impact on our knowledge of nPDFs.    

%\subsection{Drell-Yan processes}

%AFTER!

\subsection{Accessing the nuclear gluon at high-x}

One of the observables in which initial state gluons can account for most of the cross-section are jets. Unfortunately the LHC measurements in jets in $p+\mathrm{Pb}$ collisions by the ATLAS collaboration~\cite{FIXME} have been integrated over in the $0-90\%$ centrality bin instead of as the customary minimum bias data, rendering the quantity difficult to compare to in the context of collinear factorized pQCD. However, the data of the di-jets measured by the CMS collaboration were published~\cite{FIXME} and further included in the latest NLO nPDF analysis of EPPS16~\cite{FIXME}. There it was shown that, while at $Q^{2}$ where some information is lost in the PDF evolution, the di-jets have a non-negligible impact on the high-$x$ gluon distribution. As the EPPS16 fit comprises about $2000$ data points and allows for more flexible parameterizations, the effect of the di-jet data on constraining the gluon is significantly diluted. 

Nonetheless, jets remain a relevant tool to access the gluon density. In recent works~\cite{FIXME} (PHYSICAL REVIEW D 95, 094013, PHYS. REV. D 97, 114013) it has been shown that (di-)jets in $e+A$ collisions at a future Electron-Ion Collider (EIC) have the potential to reduce the theoretical uncertainties by an order of magnitude at both low and high-$x$, reaching into the anti-shadowing and EMC effect regions.

A complementary way of accessing the high-x gluon is using the charm quark structure function. This quantity is determined by tagging the charm in the final state and, theoretically, has its first non-zero contribution from the photon-gluon fusion process. In addition, this observable might hold the key to study if there is an intrinsic content of charm in the proton or nucleus or if heavy quarks appear only by radiation form the gluons. The studies of DIS reduced cross-section with simulated EIC data and its impact on the gluon nPDF~\cite{FIXME} (PRD96 (2017) no.11, 114005 show that the inclusive data could reduce the uncertainty bands up to a factor of~4 at low $x$, while the charm reduced cross-section would have a dramatic effect, diminishing the bans up to a factor of~$6\sim 8$ at high-$x$.   

%\begin{thebibliography}{5}


%\end{thebibliography}


%\end{document}



